\documentclass[10pt,a4paper]{article}
\usepackage[utf8]{inputenc}
\usepackage{amsmath}
\usepackage{amsfonts}
\usepackage{amssymb}
\usepackage{graphicx}
\usepackage[left=2cm,right=2cm,top=2cm,bottom=2cm]{geometry}
%--------------- Thanh ngang 
\usepackage{tikz}
%--------------- Chỉnh lại Header
\setlength{\headheight}{22pt}
%----------------
\usepackage{color}
\usepackage{framed}
\usepackage{fancyhdr} 
%------------------- Chia cột
\usepackage{multicol}
%-------------------Phần Header
\pagestyle{fancy}
\fancyhf{}
\rhead{\fontsize{8pt}{15pt}\selectfont \emph{ISSN (Print) : 0974-6846}\\
\vspace{-0,2cm}
\emph{ISSN (Online) : 0974-5645}}
\lhead{\fontsize{6pt}{21.3096pt}\selectfont \textbf{Indian Journal of Science and Technology}, Vol 9(11), DOI: 10.17485/ijst/2016/v9i11/88460, March 2016}
%-------------------------
%-------------------------Ngăn thụt đầu dòng
\setlength{\parindent}{0pt}
%-------------------------
\begin{document}
%-------------------------Định dạng màu
\definecolor{tieude}{rgb}{0,0.411,0.667}
%--------------------------
{\huge \fontsize{20,74pt}{10pt}\selectfont \color{tieude}\begin{flushright}
\fontsize{17,28pt}{23pt}\selectfont \textbf{Speech Emotion Recognition: Performance Analysis based on Fused Algorithms and GMM Modelling}
\end{flushright}}
\begin{flushright}
\textbf{{R. Subhashree}$^1$ and {G. N. Rathna}$^2$}
\end{flushright}
%--------------------------
\begin{flushright}
{{$^1${Department of Electrical Engineering, Amrita University, Amritanagar Post, Ettimadai,}\\
Coimbatore - 641112, Tamil Nadu, India.\\
$^2$Department of Electrical Engineering, Indian Institute of Sciences, C V Raman Ave,\\
Bengaluru - 560012, Karnataka, India;\\
z-E-mail: subha94@ymail.com, rathna@ee.iisc.ernet.in}}
\end{flushright}
%--------------------------
\vspace{-0,4cm}
\begin{tikzpicture}[remember picture,overlay]
\path[left color=white,right color=blue]
([yshift=-215pt,xshift=8pt]current page.north west)
+(18.8,-0.05pt) rectangle +(1.5cm,0.1pt);
\end{tikzpicture}
%--------------------------Box 1
\setlength{\FrameRule}{2pt}
\setlength{\FrameSep}{5pt}
\definecolor{shadecolor}{rgb}{0.929,0.949,0.976}
\begin{shaded}
{\color{tieude}{\textbf{Abstract}}}\\
%-----------
\fontsize{9pt}{13pt}\selectfont
%-----------
\textbf{Background/Objectives:} Speech emotion recognition (SER) is an important aspect of Human-Computer Interaction systems which is widely used in different sectors like healthcare, robotics, automatic call centres and distance education.
Speech emotion recognition involves in depth analysis of the signal and identifying the appropriate emotion based on
its trained database using extracted features. \textbf{Method/Statistical Analysis:} This paper aims in devising SER system
using linear prediction of the causal part of the autocorrelation sequence (OSALPC) algorithm which has been proven
to efficiently reduce noise along with Linear Frequency Cepstral Coefficients (LFCC), Linear Predictive Coding (LPC),
MFCC, LPC using cepstrum for feature extraction. After extracting the feature vectors from the voice signal, it is modelled \textbf{Findings:}
Performance was analysed and our proposed system showed an overall efficiency of 89\% when tested on German database
(Emo-DB) for 7 emotions. The overall efficiency has proven to increase compared to the studies made up to date on the
German Database. The highest emotion recognition rate was for SAD using fused algorithm which was 95.56\%. Also results
were tabulated and compared using Modified MFCC. A Graphical unit interface of the proposed system is also devised.
\textbf{Application/Improvements:} The applications of speech emotion recognition are farfetched. Further scope of this work
will be a comparison of the achieved recognition rate using algorithms with recognition rate achieved by humans.
\end{shaded}
%-----------------------------------------
\vspace{-0,7cm}
{\color{tieude}\rule{17cm}{0,1pt}}\\
{\color{tieude}\textbf{Keywords:}} Emotion Recognition, LPC, MFCC, GMM, MAP, OSALPC
%--------------------------- Văn bản 1
\begin{multicols}{2}
%--------------------------Thanh ngang 1
{\Large  \color{tieude} \rmfamily \textbf{1. Introduction}}

\begin{tikzpicture}[remember picture,overlay]
\path[left color=gray,right color=white]
([yshift=-469pt,xshift=-0.5pt]current page.north west)
+(10.5,-0.1pt) rectangle +(2cm,0.5pt);
\end{tikzpicture}
{\vspace*{-1pt}
%-------------------------
\hspace*{-10pt} Recognizing emotions from speech had diverse appli
cations in different felds ranging from health care and
medicine to entertainment. Currently it is predomi
nantly being used for facilitating interaction of humans
with computers and robots. Such robots are used in
medicine for purposes like monitoring autistic people
and helping them interpret expressions and developing
a system for therapy and counselling, or in a brain-com
puter interface to interpret emotion from voice along
with facial expressions and EEG signals to help patients
deal with anxiety, stress and depression. It can also be
used in the feld of entertainment to identify the emo
tions and reactions of users. Various algorithms have
been deployed over the years for the two important pro
cesses involved in SER (Speech Emotion Recognition)
namely feature extraction and the decision making.
Some of the algorithms are Artifcial neural networks$^1$,
combination of linear prediction cepstrum coefcients
(LPCC), Mel Frequency cepstrum coefcients (MFCC),
Linear Prediction coefcients and Mel cepstrum coef
fcients (LPCMCC) and the Support Vector Machine
(SVM); combination of HMM and SVM, ANN, KNN,
MLBetc$^2$; MFCC, Discrete wavelet transforms with
SVM, Feature extraction algorithm based on Field}
\end{multicols}
{\color{tieude}\ \rule{3cm}{0.1pt}}\\
$^*$Author for correspondence
\end{document}
